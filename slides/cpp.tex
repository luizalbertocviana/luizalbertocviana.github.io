% Created 2020-06-08 seg 18:46
% Intended LaTeX compiler: pdflatex
\documentclass[presentation]{beamer}
\usepackage[utf8]{inputenc}
\usepackage[T1]{fontenc}
\usepackage{graphicx}
\usepackage{grffile}
\usepackage{longtable}
\usepackage{wrapfig}
\usepackage{rotating}
\usepackage[normalem]{ulem}
\usepackage{amsmath}
\usepackage{textcomp}
\usepackage{amssymb}
\usepackage{capt-of}
\usepackage{hyperref}
\usepackage[newfloat, cache = false]{minted}
\RequirePackage{fancyvrb}
\DefineVerbatimEnvironment{verbatim}{Verbatim}{fontsize=\scriptsize}
\usetheme{default}
\author{Luiz Alberto do Carmo Viana}
\date{\today}
\title{Modern C++}
\hypersetup{
 pdfauthor={Luiz Alberto do Carmo Viana},
 pdftitle={Modern C++},
 pdfkeywords={},
 pdfsubject={},
 pdfcreator={Emacs 26.3 (Org mode 9.3.6)}, 
 pdflang={English}}
\begin{document}

\maketitle
\begin{frame}{Outline}
\tableofcontents
\end{frame}


\section{Introduction}
\label{sec:org1bcbcfd}
\begin{frame}[label={sec:org3bf9bb6},fragile]{History}
 \begin{itemize}
\item C with Classes: created by Bjarne Stroustrup at Bell Labs in 1979.
\item C++98: first standard.
\item C++03: requires that elements in \texttt{vector} are stored contiguously.
\item C++11: legacy support; big changes; modern C++.
\item C++14: better type deduction; generic lambdas; variable
templates.
\begin{itemize}
\item This is the default standard of GCC.
\end{itemize}
\item C++17: \texttt{optional}; \texttt{any}; \texttt{variant}.
\item C++20: to be finished (don't touch it, corona).
\end{itemize}
\end{frame}

\begin{frame}[label={sec:org1b69872},fragile]{Hello World}
 \begin{verbatim}
1  #include <iostream>
2  
3  int main(){
4    std::cout << "Hello World" << std::endl;
5  
6    return 0;
7  }
\end{verbatim}
\end{frame}

\begin{frame}[label={sec:org420774d}]{Overview}
\begin{itemize}
\item Extended C syntax.
\item Large language: significant amount of legacy; focus on modern
subset.
\item Nearly no overhead abstractions: as C, good mapping to hardware.
\end{itemize}
\end{frame}

\section{Features}
\label{sec:orgfd16063}
\begin{frame}[label={sec:orgefdb4dd},fragile]{Features}
 Uniform initialization \texttt{\{\}}:

\begin{verbatim}
 1  // initialization: calls constructor
 2  int a{}; // a == 0
 3  int x{1};
 4  std::string word{"apple"};
 5  // assignment: calls operator=
 6  int x = 1;
 7  std::string word = "apple"; // calls constructor, but is misleading
 8  word = "orange"; // here it does call operator=
 9  // initialization does not perform invisible conversions
10  int y = 3.5; // y == 3
11  int y{3.5}; // error
12  // container constructors may bring ambiguities
13  std::vector<int> vec{10}; // vec contains one int element of value 10
14  std::vector<int> vec(10); // vec contains ten int elements of value 0
\end{verbatim}
\end{frame}

\begin{frame}[label={sec:org1863c32},fragile]{Features}
 \begin{itemize}
\item \texttt{auto} keyword: type deduction.
\item Useful: less typing; type change propagation; generic programming.
\end{itemize}

\begin{verbatim}
1  auto a{true}; // bool
2  auto b{'a'}; // char
3  auto c{49}; // int
4  auto d{1.0}; // double
5  auto e{f(x)}; // e has the return type of f
6  
7  std::vector<int> vec{1, 2, 3};
8  auto it = std::begin(vec); // std::vector<int>::iterator
\end{verbatim}
\end{frame}

\begin{frame}[label={sec:org8e3744d},fragile]{Features}
 \begin{itemize}
\item \texttt{const}: variable that cannot be changed; function that promises
not to change anything.
\item \texttt{constexpr}: compile-time functions, variables and expressions.
\item \texttt{constexpr const} is not redundant.
\end{itemize}

\begin{verbatim}
 1  int global{3};
 2  
 3  constexpr int f(int x){
 4    return x * 2;
 5  }
 6  
 7  int g(int x){
 8    return global + x;
 9  }
10  
11  constexpr int a{f(2)}; // evaluated at compile time
12  constexpr int b{f(a)}; // ok
13  // ...
14  // what if global has changed?
15  const int x{global};
16  constexpr int c{x}; // error
17  constexpr int d{f(x)}; // error
18  constexpr int e{global}; // error
\end{verbatim}
\end{frame}

\begin{frame}[label={sec:org4abdff7},fragile]{Features}
 \begin{verbatim}
1  int arr[10]; //uninitialized array
2  int* ptr; // uninitialized pointer (dangerous)
3  int* ptr{arr + 2}; // initialized pointer
4  int x{*ptr}; // initialized with object that ptr points to
5  int& ref{x}; // reference
6  ref = 3; // x == 3
7  int y{4};
8  ref = y; // x == 4
9  int& ref2; // error: references must be initialized
\end{verbatim}

\begin{itemize}
\item References offer indirect access to a variable.
\item After initialization, a reference cannot refer to something else.
\item Useful as function arguments.
\item \texttt{const} references offer a view over a variable.
\end{itemize}
\end{frame}

\begin{frame}[label={sec:orgb0c6f70},fragile]{Features}
 \begin{itemize}
\item \texttt{nullptr}: more robust than \texttt{NULL} (\texttt{NULL} is still there, but
forget it).
\item \texttt{NULL} is just \texttt{0}.
\item \texttt{nullptr} is the only value of type \texttt{nullptr\_t}.
\item \texttt{color} is safe to call: no need to check \texttt{node} before calling
it.
\end{itemize}

\begin{verbatim}
1  Color color(const Node* node){
2    return node->color;
3  }
4  
5  Color color(nullptr_t leaf){
6    return Color::black;
7  }
\end{verbatim}
\end{frame}

\begin{frame}[label={sec:orgdd1f617},fragile]{Features}
 An \texttt{enum class} defines a scoped enumerator with a type.

\begin{verbatim}
1  enum class Color{red, black};
2  Color c{Color::red};
3  c = Color::black;
\end{verbatim}
\end{frame}

\begin{frame}[label={sec:org782925c},fragile]{Features}
 \begin{itemize}
\item \texttt{namespace} declares or extends a collection of definitions.
\item \texttt{std::string}: \texttt{string} is defined inside \texttt{namespace} \texttt{std}.
\item \texttt{namespace} is useful for avoiding name conflicts between
libraries.
\item Standard C++ definitions belong to \texttt{namespace} \texttt{std}.
\end{itemize}
\end{frame}

\begin{frame}[label={sec:org2fef684},fragile]{Features}
 Structured bindings make C++ feels modern.

\begin{verbatim}
 1  std::pair<std::string, int> attach_length(const std::string& str){
 2    return std::make_pair(str, str.size());
 3  }
 4  
 5  std::string str1{"something"};
 6  auto[str2, len] = attach_length(str1);
 7  
 8  // use them together with uniform initialization
 9  std::vector<std::pair<int, int>> vec{};
10  
11  for(int i = 1; i <= 3; ++i){
12    vec.push_back({i, i*10});
13  }
14  
15  for(auto[a, b] : vec){
16    std::cout << a << ' ' << b << std::endl;
17  }
\end{verbatim}
\end{frame}

\begin{frame}[label={sec:org9601d20},fragile]{Features}
 Move semantics: instead of copying, moves objects.

\begin{verbatim}
1  // copy semantics
2  std::string s1{"aaa"};
3  std::string s2{"bbb"};
4  s2 = s1; // s1 == "aaa", s2 == "aaa"
5  // move semantics
6  std::string s1{"aaa"};
7  std::string s2{"bbb"};
8  s2 = std::move(s1); // s1 is undefined, s2 == "aaa"
\end{verbatim}

\begin{itemize}
\item \texttt{std::move(Type)} returns a \texttt{Type\&\&} (rvalue reference).
\item It is undefined behaviour referencing a moved variable.
\item Sometimes it is okay to do that.
\item Good practice is not to touch moved things: don't take a chance.
\end{itemize}
\end{frame}

\section{Classes}
\label{sec:org6d3733f}
\begin{frame}[label={sec:org03601a1},fragile]{Classes}
 \begin{itemize}
\item \texttt{struct} and \texttt{class} are quite the same:
\begin{itemize}
\item \texttt{struct} defaults to public members;
\item \texttt{class} defaults to private members.
\end{itemize}
\item Let's focus on \texttt{class}.
\end{itemize}

\begin{verbatim}
 1  class Person{
 2    std::string name_;
 3    unsigned int age_;
 4  public:
 5    Person(std::string name, unsigned int age) : name_{name},
 6                                                 age_{age}
 7    {}
 8    const std::string& name() const{
 9      return name_;
10    }
11    unsigned int age() const{
12      return age_;
13    }
14  };
\end{verbatim}
\end{frame}

\begin{frame}[label={sec:org305c311},fragile]{Classes}
 \begin{itemize}
\item A \texttt{class A} always has a (unless deleted):
\begin{itemize}
\item default constructor: \texttt{A a\{\}};
\item copy constructor: \texttt{A a1\{args...\}; A a2\{a1\}};
\item move constructor: \texttt{A a1\{args...\}; A a2\{std::move(a1)\}};
\item copy assignment:
\begin{verbatim}
A a1{args...}, a2{args...};
a2 = a1;
\end{verbatim}
\item move assignment:
\begin{verbatim}
A a1{args...}, a2{args...};
a2 = std::move(a1);
\end{verbatim}
\item destructor:
\begin{verbatim}
{
  A a1{args...};
  ...
} // ~A() is called here
\end{verbatim}
\end{itemize}
\item If not provided, a implicitly declared implementation may be
used:
\begin{itemize}
\item for each member variable, it calls its counterpart.
\end{itemize}
\end{itemize}
\end{frame}

\begin{frame}[label={sec:org3df06de},fragile]{Classes}
 \texttt{Person p1\{...\}; Person p2\{...\};}
\begin{itemize}
\item \texttt{Person} has a implicitly declared:
\begin{itemize}
\item default constructor:
\begin{itemize}
\item \texttt{Person()} calls \texttt{name\_\{\}} and \texttt{age\_\{\}};
\item it can be called with \texttt{Person p\{\}};
\end{itemize}
\item copy constructor:
\begin{itemize}
\item \texttt{Person(const Person\& p)} calls \texttt{name\_\{p.name\_\}} and
\texttt{age\_\{p.age\_\}};
\item it can be called with \texttt{p2\{p1\}};
\end{itemize}
\item move constructor:
\begin{itemize}
\item \texttt{Person(Person\&\& p)} calls \texttt{name\_\{std::move(p.name\_)\}} and
\texttt{age\_\{std::move(p.age\_)\}};
\item it can be called with \texttt{p2\{std::move(p1)\}}.
\end{itemize}
\end{itemize}
\end{itemize}
\end{frame}

\begin{frame}[label={sec:org70a4c64},fragile]{Classes}
 \begin{itemize}
\item \texttt{Person} has a implicitly declared:
\begin{itemize}
\item copy assignment:
\begin{itemize}
\item \texttt{Person\& operator=(const Person\& p)} calls \texttt{name\_ = p.name\_}
and \texttt{age\_ = p.age\_};
\item it can be called with \texttt{p1 = p2};
\end{itemize}
\item move assignment:
\begin{itemize}
\item \texttt{Person\& operator=(Person\&\& p)} calls \texttt{name\_ =
         std::move(p.name\_)} and \texttt{age\_ = std::move(p.age\_)};
\item it can be called with \texttt{p1 = std::move(p2)}.
\end{itemize}
\end{itemize}
\end{itemize}
\end{frame}
\begin{frame}[label={sec:org17e0433},fragile]{Classes}
 On \{default, copy, move\} constructors, \{copy, move\} assignments and
destructor:
\begin{itemize}
\item It is okay for a \texttt{class A} to have all of them implicitly declared:
\begin{itemize}
\item if all members of \texttt{A} are language defined or;
\item all user defined members of \texttt{A} are not special:
\begin{itemize}
\item does \texttt{A} deal with files or network connections in a special
manner?
\end{itemize}
\end{itemize}
\item if you need to write one of them, write all of them.
\end{itemize}
\end{frame}
\section{Object Oriented Programming}
\label{sec:org67736a2}
\begin{frame}[label={sec:orgc71cd3d},fragile]{Object Oriented Programming}
 \begin{itemize}
\item Multiple inheritance:
\begin{itemize}
\item more powerful than single inheritance with interfaces;
\item also more dangerous (inheritance conflicts);
\item no \texttt{super} pointer.
\end{itemize}
\item Virtual classes: can't be instantiated.
\item Virtual methods:
\begin{itemize}
\item interface-ish;
\item a class with virtual methods is virtual as well.
\end{itemize}
\item \texttt{this} pointer.
\item \texttt{super} pointer makes no sense with multiple inheritance.
\end{itemize}
\end{frame}
\section{Templates}
\label{sec:org59adffd}
\begin{frame}[label={sec:org5867542},fragile]{Templates}
 \begin{itemize}
\item C++ metaprogramming device.
\item Seen a lot of them so far (\texttt{std::vector<int>}).
\end{itemize}

\begin{verbatim}
 1  template<typename Type>
 2  Type square(const Type& n){
 3    return n*n;
 4  }
 5  
 6  int x{2};
 7  std::cout << square(x) << std::endl; // Type is deduced to be int
 8  std::cout << square<int>(x) << std::endl; // Type is set explicitly
 9  
10  Matrix M{}; // suppose Matrix defines operator*
11  Matrix A{square(M)}; // Type is deduced to be Matrix
\end{verbatim}
\end{frame}
\begin{frame}[label={sec:orgcef9576},fragile]{Templates}
 Templates can have default values:
\begin{verbatim}
 1  template<typename Type = double>
 2  class Point{
 3    Type x_;
 4    Type y_;
 5  public:
 6    Point(Type x, Type y) : x_{x},
 7                            y_{y}
 8    {}
 9    Point operator+(const Point& p){
10      return {x_ + p.x_, y_ + p.y_};
11    }
12  };
13  
14  Point x{0.0, 0.0};
15  Point y{1.0, 1.0};
16  Point z{x + y};
\end{verbatim}
\end{frame}
\begin{frame}[label={sec:org74fcaff},fragile]{Templates}
 Beyond \texttt{typename}:
\begin{verbatim}
1  template<typename Type = double, unsigned int dimension = 3>
2  class Point{
3    std::array<Type, dimension> arr_;
4  public:
5    // ...
6  };
\end{verbatim}
\end{frame}
\begin{frame}[label={sec:org8880807},fragile]{Templates}
 \begin{itemize}
\item We have seen template functions and template classes.
\item What about template variables?
\end{itemize}
\begin{verbatim}
1  template<typename Type>
2  constexpr const Type pi{3.14159265358979323846};
3  
4  pi<float>; // a less precise pi
5  pi<double>; // a more precise pi
\end{verbatim}
\begin{itemize}
\item This is not the point of template variables.
\end{itemize}
\end{frame}
\begin{frame}[label={sec:orgdab6c76},fragile]{Templates}
 Templates can be specialized:
\begin{verbatim}
 1  template<typename Type>
 2  void read(Type& var){
 3    std::cin >> var;
 4  }
 5  
 6  template<>
 7  void read(std::string& var){
 8    std::cin >> std::ws;
 9    getline(std::cin, var);
10  }
\end{verbatim}

Devise better implementations: \texttt{std::vector<bool>} is implemented bitwise.
\end{frame}
\begin{frame}[label={sec:orga953ff2},fragile]{Templates}
 Template variables can be specialized as well.
\begin{verbatim}
1  template<typename A, typename B>
2  constexpr const bool equal{false};
3  
4  template<typename Type>
5  constexpr const bool equal<Type, Type>{true};
\end{verbatim}
Now we can specialize \texttt{read} using \texttt{equal} and \texttt{if constexpr}.
\begin{verbatim}
 1  template<typename Type>
 2  void read(Type& var){
 3    if constexpr (equal<Type, std::string>){
 4      std::cin >> std::ws;
 5      getline(std::cin, var);
 6    }
 7    else{
 8      std::cin >> var;
 9    }
10  }
\end{verbatim}
\end{frame}
\section{Memory Management}
\label{sec:orge5d502a}
\begin{frame}[label={sec:org195e402},fragile]{Memory Management}
 \begin{itemize}
\item C uses \texttt{malloc} and \texttt{free} for dynamic allocation.
\item Old C++ uses \texttt{new} and \texttt{delete}.
\item Modern C++ uses smart pointers.
\item We need to \texttt{\#include <memory>} to use them.
\end{itemize}
\end{frame}

\begin{frame}[label={sec:org2570950},fragile]{Memory Management}
 When used in a proper manner:
\begin{itemize}
\item \texttt{unique\_ptr<Type>}: no other \texttt{unique\_ptr<Type>} shares its
resource.
\item \texttt{shared\_ptr<Type>}: other \texttt{shared\_ptr<Type>} may share its
resource (reference counting).
\item \texttt{weak\_ptr<Type>}: offers access to the (possibly dead) resource of
a \texttt{shared\_ptr<Type>} (no reference counting).
\end{itemize}
\end{frame}

\begin{frame}[label={sec:org1ee7cda},fragile]{Memory Management}
 Forget about \texttt{new}:
\begin{itemize}
\item \texttt{unique\_ptr<Type>}: \texttt{make\_unique<Type>(args...)}.
\item \texttt{shared\_ptr<Type>}: initialize it with a moved \texttt{unique\_ptr<Type>}.
\item \texttt{weak\_ptr<Type>}: initialize it with a \texttt{shared\_ptr<Type>}.
\end{itemize}
Forget about \texttt{delete}:
\begin{itemize}
\item \texttt{unique\_ptr<Type>}: when it dies or is assigned, resource is
freed.
\item \texttt{shared\_ptr<Type>}: a resource is freed when its last \texttt{shared\_ptr}
is dead or assigned.
\item \texttt{weak\_ptr<Type>}: unless locked, it does not prevent a resource
from being freed (\texttt{lock} produces a \texttt{shared\_ptr} from a \texttt{weak\_ptr}).
\end{itemize}
\end{frame}

\begin{frame}[label={sec:org67822f2},fragile]{Memory Management}
 \begin{verbatim}
 1  class A{
 2    // ...
 3  };
 4  
 5  // C style
 6  A* obj = malloc(sizeof(A));
 7  // obj still needs to be initialized
 8  free(obj);
 9  // Old C++ style (at least calls a constructor)
10  A* obj = new A(...);
11  delete obj;
12  // C++11 style
13  std::unique_ptr<A> obj{new A(...)};
14  // deletion is automatic (end of scope)
15  // C++14 style
16  std::unique_ptr<A> obj{std::make_unique(...)}; // or
17  auto obj{std::make_unique<A>(...)};
18  // deletion is automatic (end of scope)
19  // shared pointer
20  std::shared_ptr<A> shared_obj{std::move(obj)};
21  // deletion is automatic (reference counting)
22  // weak pointer
23  std::weak_ptr<A> weak_obj{obj};
24  // weak_ptr cannot delete obj
\end{verbatim}
\end{frame}
\begin{frame}[label={sec:org59f1241},fragile]{Memory Management}
 Allocating contiguous memory
\begin{verbatim}
 1  class A{
 2    // ...
 3  };
 4  
 5  // C style stack memory (size already known)
 6  A arrA[100];
 7  
 8  // C style heap memory (size not known)
 9  int n = // ...
10  A* arrA = malloc(n * sizeof(A));
11  
12  // C++ style stack memory
13  std::array<A, 100> arrA{};
14  
15  // C++ style heap memory
16  int n{...};
17  std::vector<A> arrA{};
18  arrA.reserve(n);
\end{verbatim}
\begin{itemize}
\item GCC \texttt{push\_back} doubles \texttt{vector} capacity when it is full.
\item \(n\) \texttt{push\_back} operations on empty \texttt{vector}:
\begin{itemize}
\item \(\frac{n}{2} + \frac{n}{4} + \dots + 1 = 2n - 1\);
\item after \texttt{reserve(n)}, they cost \(n\).
\end{itemize}
\end{itemize}
\end{frame}
\section{Lambda Functions}
\label{sec:org4d3705c}
\begin{frame}[label={sec:org5674657},fragile]{Lambda Functions}
 Anonymous functions.
\begin{verbatim}
 1  const auto f{[](int& x) {x += 3;}};
 2  std::vector<int> vec{1, 2, 3, 4, 5};
 3  for (auto& e : vec){
 4    f(e);
 5  }
 6  // {4, 5, 6, 7, 8}
 7  int sum{0};
 8  const auto g{[&sum](int x) {sum += x;}};
 9  for (auto e : vec){
10    g(e);
11  }
12  // sum == 30
\end{verbatim}
\end{frame}
\section{Iterators}
\label{sec:orgcc33bf6}
\begin{frame}[label={sec:org7c34ac7},fragile]{Iterators}
 Let's perform a traversal using a pointer.
\begin{verbatim}
 1  // C style
 2  int arr[10];
 3  
 4  int* begin = arr;
 5  int* end = arr + 10; // pointer arithmetic: arr + sizeof(int) * 10
 6  
 7  int* it;
 8  for(it = begin; it != end; ++it){
 9    // do something with *it
10    *it += 3;
11  }
\end{verbatim}
Iterators provide a high level pointer arithmetic abstraction.
\end{frame}
\begin{frame}[label={sec:org322d9a7},fragile]{Iterators}
 We use \texttt{std::begin} and \texttt{std::end} to control a traversal.
\begin{verbatim}
1  template<typename Collection>
2  void increment_all(Collection& col){
3    for (auto it{std::begin(col)}; it != std::end(col); ++it){
4      *it = *it + 1;
5    }
6  }
\end{verbatim}
\texttt{increment\_all} can be used with any container that provides iterators.
\begin{verbatim}
1  std::array<int, 5> arr{1, 2, 3, 4, 5};
2  increment_all(arr); // {2, 3, 4, 5, 6}
3  std::vector<int> vec{1, 2, 3, 4, 5};
4  increment_all(vec); // {2, 3, 4, 5, 6}
5  std::string str{"apple"};
6  increment_all(str); // "bqqmf"
7  // ugly C arrays
8  int ugly[5]{1, 2, 3, 4, 5};
9  increment_all(ugly); // {2, 3, 4, 5, 6}
\end{verbatim}
Also works with \texttt{std::list}, \texttt{std::deque} and some others.
\end{frame}
\begin{frame}[label={sec:orgb79c1f9},fragile]{Iterators}
 Another example.
\begin{verbatim}
 1  template<typename Iterator, typename Function>
 2  void apply_to(Iterator begin, Iterator end, Function f){
 3    for (auto it{begin}; it != end; ++it){
 4      *it = f(*it);
 5    }
 6  }
 7  
 8  std::array<int, 5> arr{1, 2, 3, 4, 5};
 9  apply_to(std::begin(arr) + 1, std::begin(arr) + 3,
10           [](int x) {return x + 3;});
11  // {1, 5, 6, 4, 5}
\end{verbatim}
\end{frame}
\begin{frame}[label={sec:org77d993c},fragile]{Iterators}
 Template variables love lambda functions.
\begin{verbatim}
1  template<typename Type, Type t>
2  constexpr const auto add{[](Type x) {return x + t;}};
3  
4  std::array<int, 5> arr{1, 2, 3, 4, 5};
5  apply_to(std::begin(arr) + 2, std::end(arr), add<int, 5>);
6  // {1, 2, 8, 9, 10}
\end{verbatim}
\end{frame}
\begin{frame}[label={sec:orgb1a5ca7},fragile]{Iterators}
 \begin{center}
\begin{tabular}{lll}
\texttt{std::} & Read-only & Reverse\\
\hline
\texttt{begin}, \texttt{end} & no & no\\
\texttt{cbegin}, \texttt{cend} & yes & no\\
\texttt{rbegin}, \texttt{rend} & no & yes\\
\texttt{crbegin}, \texttt{crend} & yes & yes\\
\end{tabular}
\end{center}

\begin{itemize}
\item Read about iterator adaptors.
\begin{itemize}
\item We are going to use \texttt{back\_inserter} here.
\end{itemize}
\item Why bother with iterators?
\begin{itemize}
\item generic algorithms; STL.
\end{itemize}
\end{itemize}
\end{frame}
\begin{frame}[label={sec:orge94ca67},fragile]{Iterators}
 Example of a (silly) iterator adaptor.
\begin{verbatim}
 1  template<typename Collection>
 2  class Reverse{
 3    Collection& col_;
 4  public:
 5    Reverse(Collection& col) : col_{col} {}
 6    using iterator          = typename Collection::reverse_iterator;
 7    using traits            = std::iterator_traits<iterator>;
 8    using difference_type   = typename traits::difference_type;
 9    using value_type        = typename traits::value_type;
10    using reference         = typename traits::reference;
11    using pointer           = typename traits::pointer;
12    using iterator_category = typename traits::iterator_category;
13    iterator begin() {return std::rbegin(col_);}
14    iterator end()   {return std::rend(col_);}
15  };
\end{verbatim}
\end{frame}
\begin{frame}[label={sec:orgdb8a5e8},fragile]{Iterators}
 \begin{verbatim}
 1  template<typename Type>
 2  class Range{
 3    Type from_, to_, step_;
 4  public:
 5    Range(Type from, Type to, Type step) : from_{from}, to_{to}, step_{step} {}
 6    Range(Type from, Type to) : Range{from, to, 1} {}
 7    struct iterator{
 8      Type x_; Range& parent_;
 9      using difference_type = Type; using value_type = Type;
10      using reference = Type&; using pointer = Type*;
11      using iterator_category = std::input_iterator_tag;
12      iterator(Type x, Range& parent) : x_{x}, parent_{parent} {}
13      bool operator!=(const iterator& it) { return x_ != it.x_;}
14      bool operator==(const iterator& it) {return x_ == it.x_;}
15      reference operator*() {return x_;}
16      difference_type operator-(const iterator& it) {return x_ - it.x_;}
17      iterator& operator++(){
18        Type next{x_ + parent_.step_}, to{parent_.to_};
19        x_ = next < to ? possible_step : to;
20        return *this;
21      }
22    };
23    iterator begin() {return iterator{from_, *this};}
24    iterator end() {return iterator{to_, *this};}
25  };
\end{verbatim}
\end{frame}
\section{Standard Template Library}
\label{sec:org7395937}
\begin{frame}[label={sec:org35749c9}]{Standard Tenplate Library}
\begin{itemize}
\item C++ Standard Library.
\item Containers, algorithms and other facilities.
\begin{itemize}
\item General interfaces, some specialized implementations.
\end{itemize}
\item Let's see some of them by example.
\end{itemize}
\end{frame}
\begin{frame}[label={sec:orgaacebae},fragile]{Standard Template Library}
 \begin{verbatim}
 1  #include <algorithm>
 2  
 3  template<typename Type>
 4  constexpr const auto is_odd{[](Type x) {return x % 2 == 1;}};
 5  
 6  template<typename Type, Type ub>
 7  constexpr const auto leq{[](Type x) {return x <= ub;}};
 8  
 9  std::vector<int> vec{1, 3, 5, 7, 9};
10  std::all_of(std::cbegin(vec), std::cend(vec), is_odd<int>); // 1
11  std::vector<int> vec{1, 3, 6, 7, 9};
12  std::all_of(std::cbegin(vec), std::cend(vec), is_odd<int>); // 0
13  
14  std::any_of(std::cbegin(vec), std::cend(vec), leq<int, 1>); // 1
15  std::any_of(std::cbegin(vec), std::cend(vec), leq<int, 0>); // 0
\end{verbatim}
\end{frame}
\begin{frame}[label={sec:org1641f89},fragile]{Standard Template Library}
 \begin{verbatim}
 1  #include <algorithm>
 2  
 3  template<typename Type>
 4  class Leq{
 5    Type ub_;
 6  public:
 7    Leq(Type ub) : ub_{ub} {}
 8    bool operator()(Type x){
 9      return x <= ub_;
10    }
11  };
12  
13  int n{1};
14  std::any_of(std::cbegin(vec), std::cend(vec), Leq{n}); // 1
15  n = 0;
16  std::any_of(std::cbegin(vec), std::cend(vec), Leq{n}); // 0
\end{verbatim}
\end{frame}
\begin{frame}[label={sec:org6801dfe},fragile]{Standard Template Library}
 \texttt{\#include <algorithm>}
\begin{center}
\begin{tabular}{ll}
Function & Description\\
\hline
\texttt{none\_of} & 1 iff no element of iterator range satisfies predicate\\
\texttt{for\_each} & applies function to each element of iterator range\\
\texttt{find} & iterator to first occurrence of element in range\\
\texttt{find\_if} & first element of range satisfying predicate\\
\texttt{find\_end} & searches \texttt{range1} for last occurrence of \texttt{range2}\\
\texttt{find\_first\_of} & first element in \texttt{range1} occurring in \texttt{range2}\\
\texttt{adjacent\_find} & first adjacent elements satisfying binary predicate\\
\texttt{count} & number of occurrences of element in range\\
\texttt{count\_if} & number of elements in range satisfying predicate\\
\texttt{mismatch} & pair of first mismatching positions of ranges\\
\texttt{equal} & 1 iff two ranges are equal\\
\texttt{is\_permutation} & 1 iff \texttt{range1} is permutation of \texttt{range2}\\
\texttt{search} & searches \texttt{range1} for first occurrence of \texttt{range2}\\
\end{tabular}
\end{center}
\end{frame}
\begin{frame}[label={sec:orgfd326c7},fragile]{Standard Template Library}
 \begin{verbatim}
 1  std::vector<int> vec1{1, 2, 3, 4, 5, 6, 7, 8, 9, 10};
 2  std::vector<int> vec2{10, 9, 8, 7, 6, 5, 4, 3, 2, 1};
 3  
 4  std::count_if(std::cbegin(vec1), std::cend(vec1), is_odd<int>); // 5
 5  
 6  auto begin1{std::cbegin(vec1)};
 7  auto end1{std::cend(vec1)};
 8  auto begin2{std::cbegin(vec2)};
 9  auto end2{std::cend(vec2)};
10  
11  std::distance(begin1, end1) == std::distance(begin2, end2) &&
12    std::is_permutation(begin1, end1, begin2);  // 1
\end{verbatim}
\end{frame}
\begin{frame}[label={sec:org5c5e421},fragile]{Standard Template Library}
 \begin{verbatim}
 1  bool is_prime(unsigned x){
 2    if (x == 1){
 3      return false;
 4    }
 5    if (x == 2 || x == 3){
 6      return true;
 7    }
 8    if (x % 2 == 0){
 9      return false;
10    }
11    else{
12      unsigned ub{static_cast<unsigned>(floor(sqrt(x)))};
13      Range<unsigned> divisor_candidates{3, ub + 1, 2};
14  
15      return std::none_of(std::cbegin(divisor_candidates),
16                         std::cend(divisor_candidates),
17                         [x](unsigned d) {return x % d == 0;});
18    }
19  }
\end{verbatim}
\end{frame}
\begin{frame}[label={sec:org6be59cf},fragile]{Standard Template Library}
 \begin{center}
\begin{tabular}{ll}
Function & Description\\
\hline
\texttt{copy} & copies range to output iterator\\
\texttt{copy\_if} & copies range elements satisfying predicate to output\\
\texttt{swap} & exchanges contents of objects\\
\texttt{swap\_ranges} & applies \texttt{swap} to two ranges\\
\texttt{transform} & operator applied to one (two) range(s) goes to output\\
\texttt{replace} & overwrites \texttt{value1} with \texttt{value2} in range\\
\texttt{replace\_if} & assigns value to elements of range satisfying predicate\\
\texttt{generate} & fills range with return values of function\\
\texttt{remove} & removes value from range\\
\texttt{remove\_if} & removes elements from range satisfying predicate\\
\texttt{unique} & removes consecutive duplicates from range\\
\end{tabular}
\end{center}
Note: \texttt{remove} and \texttt{remove\_if} do not realloacte memory.
\end{frame}
\begin{frame}[label={sec:org31dd10d},fragile]{Standard Template Library}
 \begin{verbatim}
 1  std::vector<int> vec{1, 2, 3, 4, 5, 6, 7, 8, 9, 10};
 2  
 3  // this does not alter vec capacity
 4  auto new_end{std::remove_if(std::begin(vec), std::end(vec), is_prime)};
 5  // now, vec.erase(new_end, std::end(vec)) would alter vec capacity
 6  
 7  std::vector<int> vec2{};
 8  vec2.reserve(std::distance(std::begin(vec), new_end));
 9  
10  std::copy(std::begin(vec), new_end, std::back_inserter(vec2));
11  
12  for (auto e : Reverse{vec2}){
13    std::cout << e << ' ';
14  }
15  // 10 9 8 6 4 1 
\end{verbatim}
\end{frame}
\begin{frame}[label={sec:org56751b4},fragile]{Standard Template Library}
 \texttt{<algorithm>} also has functions related to:
\begin{itemize}
\item partition;
\item sorting;
\item binary search (on sorted ranges);
\item merging (of sorted ranges);
\item heap adaptors;
\item minimum and maximum values;
\end{itemize}
Also take a look at:
\begin{itemize}
\item containers: \texttt{<set>}, \texttt{<map>}, \texttt{<bitset>};
\item container adaptors: \texttt{<stack>}, \texttt{<queue>};
\item utilities: \texttt{<optional>}, \texttt{<functional>}, \texttt{<tuple>}, \texttt{<type\_traits>}.
\end{itemize}
\end{frame}
\section{Further Reading}
\label{sec:org6f5532e}
\begin{frame}[label={sec:org2b7cebe}]{Further Reading}
\begin{itemize}
\item A Tour of C++, by Bjarne Stroustrup.
\item C++ Core Guidelines (\href{https://isocpp.github.io/CppCoreGuidelines/CppCoreGuidelines}{link}), by Bjarne Stroustrup and Herb Sutter.
\item C++17 STL Cookbook, by Jacek Galowicz.
\end{itemize}
\end{frame}

\begin{frame}[label={sec:org31c6ee1}]{Thank you}
\begin{center}
\Huge{Thank you!}
\end{center}
\end{frame}
\end{document}